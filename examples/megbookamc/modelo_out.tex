 

\documentclass{article}

\usepackage{tikz}
\usetikzlibrary{arrows}

\usepackage[utf8]{inputenc}

\begin{document}




% ==============================================
%
% meg.new( "E97G40_produto_escalar2_R2_008", ekey = 10 )
%
% ==============================================



Na figura estão representados dois vetores, $\overrightarrow{AD}$ e $\overrightarrow{AE}$, de normas $8$ e $6$, respetivamente.

\begin{tikzpicture}[line cap=round,line join=round,>=triangle 45,x=1.0cm,y=1.0cm]
\clip(-0.56,-0.74) rectangle (8.34464540553,2.06892908111);
\draw [->] (0.0,-0.0) -- (6,0.0);
\draw [->] (0.0,-0.0) -- (7.84464540553,1.56892908111);
\draw (5,1)-- (5,0.0);
\draw (0,0)-- (5,1);   
\draw (0,0)-- (5,0);
\draw (0,0) node[anchor=north] {$A$};
\draw (5,0) node[anchor=north] {$C$};
\draw (6,0.0) node[anchor=north] {$E$};
\draw (5,1) node[anchor=south east] {$B$};
\draw (7.84464540553,1.56892908111) node[anchor=south] {$D$};
\end{tikzpicture}

No segmento de reta $[AD]$ está assinalado um ponto $B$.

No segmento de reta $[AE]$ está assinalado um ponto $C$.

O triângulo $[ABC]$ é retângulo em $C$ e $\overline{AC}=5$ e $\overline{CB}=1$.


Indique o valor do produto escalar $\overrightarrow{AD} \cdot \overrightarrow{AE}$.

    



\begin{itemize}

% wrong
\item 

$$ \overrightarrow{AD} \cdot \overrightarrow{AE}=240 $$

  

% correct
\item 
$$  \overrightarrow{AD} \cdot \overrightarrow{AE}=\frac{120}{13} \, \sqrt{26}  $$


% wrong
\item 

$$ \overrightarrow{AD} \cdot \overrightarrow{AE}=\frac{24}{13} \, \sqrt{26} $$



% wrong
\item 

$$ \overrightarrow{AD} \cdot \overrightarrow{AE}=\frac{48}{5} $$



\end{itemize}

\noindent\textbf{Resolução}

Resolução:
Por definição de produto escalar, sabe-se que
$$ \overrightarrow{AD} \cdot \overrightarrow{AE}=||\overrightarrow{AD}|| \, ||\overrightarrow{AE}|| \, \cos\alpha,$$
em que $\alpha$ é o ângulo formado pelos dois vetores.<p>
Uma vez que é dada a norma de cada um dos vetores, é necessária a determinação do valor de $\cos\alpha$ para posterior cálculo do produto escalar.<p>
Num triângulo retângulo, o cosseno de um ângulo agudo pode ser determinado pela razão entre a medida do cateto que lhe é adjacente e a medida da hipotenusa. Sendo $[ABC]$ um triângulo retângulo, para $\alpha=\angle CAB$, tem-se que $\displaystyle \cos \alpha =\frac{\overline{AC}}{\overline{AB}}$.<p> Determine-se, aplicando o Teorema de Pitágoras, $\overline{AB}$:
    $$\overline{AB}^2=\overline{AC}^2+\overline{BC}^2=5^2+1^2= 26.$$
Sendo $\overline{AB}$ a medida do segmento de reta $[AB]$, tem-se que $\overline{AB}=\sqrt{26}=\sqrt{26}$.<p>

Pode, agora, determinar-se o valor de $\cos\alpha$:
    $$\cos\alpha=\frac{5}{\sqrt{26}}=\frac{5}{26} \, \sqrt{26}.$$
Assim,


\begin{eqnarray*}
    \overrightarrow{AD} \cdot \overrightarrow{AE}&=& ||\overrightarrow{AD}|| \,||\overrightarrow{AE}|| \, \cos\alpha \\
                                           &=& 8 \times 6\times \frac{5}{26} \, \sqrt{26} \\
                                           &=& \frac{120}{13} \, \sqrt{26}. \\
\end{eqnarray*} 

A resposta correta é assim $\displaystyle \frac{120}{13} \, \sqrt{26}$.


%Fim do "E97G40_produto_escalar2_R2_008"




\textbf{Name:}~\verb+E97G40_soma_001+

\noindent\textbf{Sumário} 



uma soma

Palavras chave: Produto escalar

Autor: Ana Palmeira, 2014




\noindent\textbf{Problema} 




Calcule: $a1 + a2 = $    



\noindent\textbf{Resolução}



$a1 + a2 = res$   pois ....




\noindent\textbf{Código Sage/Python}

\begin{verbatim}
class E97G40_soma_001(Exercise):
   
    def make_random(s):
        
        s.a1=ur.iunif(1,5)
        s.a2=ur.iunif(1,5)
                            
            
    def solve(s):
        
        s.res=s.a1 + s.a2
        

\end{verbatim}

\noindent\textbf{Problema (Exemplo) ekey= 10 }




Calcule: $5 + 1 = $    



\noindent\textbf{Resolução (Exemplo)}



$5 + 1 = 6$   pois ....








\end{document}
