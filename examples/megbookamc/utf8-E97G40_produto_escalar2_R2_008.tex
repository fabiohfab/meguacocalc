

\documentclass{article}



%
% common.tex: common packages and declarations.
%
% MEGUA package for Sage Math
% University of Aveiro @ 2012
% http://cms.ua.pt/megua
%

% =====
% Fonts
% =====
\usepackage{amsmath}
\usepackage{amsfonts}
\usepackage{amssymb}
\usepackage{euro}


% ============
% if then else
% ============
\usepackage{etoolbox}


% =============
% Graphics
% =============
\usepackage{pgf,tikz}
\usetikzlibrary{arrows}
\usetikzlibrary{shapes,backgrounds}
\usetikzlibrary{datavisualization}
\usetikzlibrary{datavisualization.formats.functions}
\usetikzlibrary{matrix}


\usepackage{tikz-3dplot}
%TODO: Removed until proved otherwise
%\usepackage[pdftex]{graphicx} 


% =================
% Language settings
% =================
%\usepackage[utf8]{inputenc}
\usepackage[latin1]{inputenc}
\usepackage[portuges]{babel}
\usepackage[T1]{fontenc} %TODO: what is T1 for ???


\let\tan\undefined
\DeclareMathOperator{\tan}{tg}
\let\sin\undefined
\DeclareMathOperator{\sin}{sen}
%\let\log\undefined
%\DeclareMathOperator*{\log}{ln}




\pagestyle{empty}


\oddsidemargin 0.0in
\evensidemargin 0.0in
\textwidth 6.45in
\topmargin 0.0in
\headheight 0.0in
\headsep 0.0in
\textheight 9.0in


\begin{document}

{\centering\bf \verb"E97G40_produto_escalar2_R2_008" }

\noindent\textbf{Sumário}


Neste exercício pretende determinar-se o produto escalar entre dois vetores, dadas as suas normas, com dados no enunciado que possibilitam a determinação do cosseno do ângulo por eles formado. Tanto as normas dos vetores, como as medidas dos catetos, são valores parametrizados. 
A figura foi inicialmente criada utilizando o software "Geogebra", exportada para linguagem "TikZ" e posteriormente parametrizada.
Neste exercício houve necessidade de efetuar arredondamentos de números, utilizando instrução "round", para a parametrização da figura.



Palavras chave: Produto escalar


SIACUAstart
level=1;  slip= 0.2; guess=0.25; discr = 0.3 
concepts = [(4341, 1)]
SIACUAend

Autor: Ana Palmeira, 2014




\noindent\textbf{Problema ekey=1440587254 }



Na figura estão representados dois vetores, $\overrightarrow{AD}$ e $\overrightarrow{AE}$, de normas $5$ e $5$, respetivamente.

\begin{tikzpicture}[line cap=round,line join=round,>=triangle 45,x=1.0cm,y=1.0cm]
\clip(-0.56,-0.74) rectangle (5.50000000000000,4.972135955);
\draw [->] (0.0,-0.0) -- (5,0.0);
\draw [->] (0.0,-0.0) -- (2.2360679775,4.472135955);
\draw (2,4)-- (2,0.0);
\draw (0,0)-- (2,4);   
\draw (0,0)-- (2,0);
\draw (0,0) node[anchor=north] {$A$};
\draw (2,0) node[anchor=north] {$C$};
\draw (5,0.0) node[anchor=north] {$E$};
\draw (2,4) node[anchor=south east] {$B$};
\draw (2.2360679775,4.472135955) node[anchor=south] {$D$};
\end{tikzpicture}

No segmento de reta $[AD]$ está assinalado um ponto $B$.

No segmento de reta $[AE]$ está assinalado um ponto $C$.

O triângulo $[ABC]$ é retângulo em $C$ e $\overline{AC}=2$ e $\overline{CB}=4$.


Indique o valor do produto escalar $\overrightarrow{AD} \cdot \overrightarrow{AE}$.

    

 

\noindent\textbf{Resolução}



<multiplechoice>

<choice>
$$  \overrightarrow{AD} \cdot \overrightarrow{AE}=5 \, \sqrt{5}  $$
</choice>

<choice>

$$ \overrightarrow{AD} \cdot \overrightarrow{AE}=10 \, \sqrt{5} $$

</choice>

<choice>

$$ \overrightarrow{AD} \cdot \overrightarrow{AE}=\frac{25}{2} $$

</choice>

<choice>

$$ \overrightarrow{AD} \cdot \overrightarrow{AE}=50 $$

</choice>

</multiplechoice>


Resolução:
Por definição de produto escalar, sabe-se que
$$ \overrightarrow{AD} \cdot \overrightarrow{AE}=||\overrightarrow{AD}|| \, ||\overrightarrow{AE}|| \, \cos\alpha,$$
em que $\alpha$ é o ângulo formado pelos dois vetores.<p>
Uma vez que é dada a norma de cada um dos vetores, é necessária a determinação do valor de $\cos\alpha$ para posterior cálculo do produto escalar.<p>
Num triângulo retângulo, o cosseno de um ângulo agudo pode ser determinado pela razão entre a medida do cateto que lhe é adjacente e a medida da hipotenusa. Sendo $[ABC]$ um triângulo retângulo, para $\alpha=\angle CAB$, tem-se que $\displaystyle \cos \alpha =\frac{\overline{AC}}{\overline{AB}}$.<p> Determine-se, aplicando o Teorema de Pitágoras, $\overline{AB}$:
    $$\overline{AB}^2=\overline{AC}^2+\overline{BC}^2=2^2+4^2= 20.$$
Sendo $\overline{AB}$ a medida do segmento de reta $[AB]$, tem-se que $\overline{AB}=\sqrt{20}=2 \, \sqrt{5}$.<p>

Pode, agora, determinar-se o valor de $\cos\alpha$:
    $$\cos\alpha=\frac{2}{2 \, \sqrt{5}}=\frac{1}{5} \, \sqrt{5}.$$
Assim,


\begin{eqnarray*}
    \overrightarrow{AD} \cdot \overrightarrow{AE}&=& ||\overrightarrow{AD}|| \,||\overrightarrow{AE}|| \, \cos\alpha \\
                                           &=& 5 \times 5\times \frac{1}{5} \, \sqrt{5} \\
                                           &=& 5 \, \sqrt{5}. \\
\end{eqnarray*} 

A resposta correta é assim $\displaystyle 5 \, \sqrt{5}$.






\end{document}

