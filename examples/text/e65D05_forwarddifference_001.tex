%PROBLEM

%marcas: interpolação, inteiros
%tags: interpolation

Seja $p_2$ o polinómio interpolador com base no suporte:
\[
\begin{array}{c|cccc}
x & inx0 & inx1 & inx2  \\ \hline
y & iny0 & iny1 & iny2  \\
\end{array}
\]
Determine o valor de $p_2(inxv)$ usando o método das diferenças progressivas?


%ANSWER

Usando diferenças progressivas:
\[
\begin{array}{c|ccc}
x & y  & \Delta_1(f(\cdot)) & \Delta_1^2(f(x_0)) \\ \hline
inx0 & iny0 \\
   &    & ondp1 \\
inx1  & iny1  &    &  ondp3 \\
   &    & ondp2 \\
inx2  & iny2
\end{array}
\]
Sendo $h=inh$, temos, pelas diferenças divididas, 
\begin{eqnarray*}
 p_2(x=inxv) & = & y_0 + \frac{\Delta_1}{1! h} (x-x_0) + \frac{\Delta_2}{2! h^2} (x-x_0)(x-x_1) \\
            & = & iny0 + \frac{ondp1}{1! inh} (x-onx0) + \frac{ondp3}{2! inh^2} (x-onx0)(x-onx1) \\
           & = & onresultexact\simeq onresultapprox$.
\end{eqnarray*}





